\documentclass{article}

\usepackage{xepersian}
\usepackage{fontspec}

\usepackage{amsmath}
\usepackage{amssymb}

\settextfont{XB Niloofar}
\newcommand{\tildevar}{\mathord{\sim}}

\begin{document}
	\section*{ جواب تمرینات فصل دوم}
		\subsection*{جواب سوال اول :}
			\subsubsection*{(i)}
				حل با جدول ارزش ها

			\subsubsection*{(ii)}
				برای اثبات $\equiv p$ $p \land (q \lor p)$، یک $\land (q \lor\tildevar q)$ اضافه میکنیم. این کار مشکلی ندارد چون ارزش این گزاره به خودی خود نادرست است که اضافه کردن آن در گزاره  $p \land (q \lor p)$ تاثیری ندارد. 
				\\  برهان : \\
				\[
				p \land (p \lor q) \equiv p \land (p \lor q) \land (q \lor \tildevar q) \equiv p \land [q \lor (p \land \tildevar q) \rightarrow]
				\]
				\[
				(p \land q) \lor (p \land \tildevar q) \equiv p \land (q \lor \tildevar q) \equiv p
				\]
				اثبات دیگری به همین روش
				
		\subsection*{جواب سوال دوم :}
			\subsubsection*{(i)}
				\[ [p \land (p \Rightarrow q)] \Rightarrow q \rightarrow \tildevar[p \land (\tildevar p \lor q)] \lor q \rightarrow \tildevar[(p \land 									\tildevar p) \lor (p \land q)] \lor q \equiv\tildevar p \lor \tildevar q \lor q \equiv T\]

			\subsubsection*{(ii)}
				\[ (\tildevar p \lor q) \Leftrightarrow (p \Rightarrow q) \rightarrow [(\tildevar p \lor q) \Rightarrow (p \Rightarrow q)] \land [(p \Rightarrow q) 						\Rightarrow (\tildevar p \lor q)] \rightarrow  \]
				\[
				 [\tildevar (\tildevar p \lor q) \lor (\tildevar p \lor q)] \land [\tildevar (\tildevar p \lor q) \lor (\tildevar p \lor q)] 	\rightarrow 
				 (p \land \tildevar q) \lor (\tildevar p \lor q) \rightarrow q \lor [\tildevar p \lor (p \land \tildevar q)]
				\]
				\[
				\rightarrow q \lor (\tildevar q \lor \tildevar p) \equiv T
				\]

			\subsubsection*{(iii)}
				\[
				(p \lor q) \lor (\tildevar p) \Rightarrow (q \land \tildevar p) \rightarrow [\tildevar p \lor p \lor q] \lor (q \land \tildevar p) \rightarrow
				F \lor (q \land \tildevar p) \equiv q \land \tildevar p
				\]
				در نتیجه این گزاره همیشه درست نیست و ارزش آن مساوی $q$$\land$$\tildevar p$ است.

		\subsection*{جواب سوال سوم:}
			\subsubsection*{(i)}
				اثبات با جدول ارزش ها

			\subsubsection*{(ii)}
				\[
				p*p \equiv (\tildevar p) \rightarrow p*p \equiv (\tildevar p) \lor (\tildevar p) \equiv \tildevar p
				\]
			
			\subsubsection*{(iii)}
				\[
				(p \land q) \equiv (p*q)*(q*p) \rightarrow (p*q)*(q*p) \equiv (\tildevar p \lor \tildevar q) * (\tildevar q \lor \tildevar p) \rightarrow
				\]
				\[
				[\tildevar (\tildevar p \lor \tildevar q)] \lor [\tildevar (\tildevar q \lor \tildevar p)] \equiv (p \land q) \lor (q \land p) \equiv p \land q
				\]

			\subsubsection*{(iv)}
				\[
					p*(q*q) \equiv (p \Rightarrow q) \rightarrow q*q \equiv \tildevar q \lor \tildevar q \equiv \tildevar q \rightarrow p*(q*q) \equiv p*\tildevar q
				\]
				\[
					\rightarrow p*\tildevar q \equiv \tildevar p \lor \tildevar (\tildevar q) \equiv \tildevar p \lor q \equiv p \Rightarrow q
				\]
			
			\subsubsection*{(v)}
				\[
					(p*p)*(q*q) \equiv p \lor q \rightarrow p*p \equiv \tildevar p , q*q \equiv \tildevar q \rightarrow
				\]
				\[
					(p*p)*(q*q) \equiv \tildevar (\tildevar p) \lor \tildevar (\tildevar q) \equiv p \lor q
				\]

		\subsection*{جواب سوال چهارم :}
			\subsubsection*{(i)}
				\[
				p \land q, p \Rightarrow \tildevar q \mapsto \tildevar q
				\]
				برای معتبر بودن این گزاره، فرض میگیریم $p \land q$ درست است پس، ارزش $p$ و $q$ درست است.\\
				وقتی که $p$ و $q$ درست باشند آنگاه، $p \Rightarrow \tildevar q$ , $\tildevar q$ همگی نادرست هستند پس بحث بالا معتبر است.
			
			\subsubsection*{(ii)}
				فرض میکنیم ارزش$p \land \tildevar q$ درست است. پس ارزش $p$, $q$ به ترتیب درست و نادرست است.\\
				از آنجا که $p$ درست و $q$ نادرست است پس $q \Rightarrow p$, $\tildevar p$ به ترتیب درست و نادرستند پس بحث بالا نامعتبر است.

			\subsubsection*{(iii)}
				\[
				\tildevar p_a \mapsto \tildevar \forall x:  p_x
				\]
				با برهان خلف فرض میکنیم $\tildevar \forall x:  p_x$ نادرست است پس به ازای همه $x$ ها، $p_x$ درست است پس $p_a$ درست و نقیض آن نادرست خواهد بود پس بحث معبتر است.
			
			\subsubsection*{(iv)}
				\[
				\forall x: (p_x \Rightarrow q_x), \tildevar q_a \mapsto \tildevar \forall x: p_x
				\]
				فرض میکنیم $\tildevar q_a$ درست است پس $q_a$ نادرست خواهد بود. پس برای درست بودن ارزش $\forall x: (p_x \Rightarrow q_x)$،\\
				$\forall x : p_x$ نادرست خواهد بود. آنگاه نقیض آن درست خواهد بود که معلوم میشود بحث معبتر است.

			\subsubsection*{(v)}
				با برهان خلف فرض میکنیم $\forall x: r_x$ نادرست است پس برای درست بودن $\forall x: (q_x \Rightarrow r_x)$ و $\forall x: (p_x \Rightarrow r_x)$،
				هر دو $\forall x: p_x$ و $\forall x: q_x$ نادرست خواهند بود پس $\forall x: q_x \lor p_x$ نادرست خواهند بود. پس بحث معتبر است.
		
		\subsection*{جواب سوال پنجم :}
			\subsubsection*{(i)}
				مثال نقض :
				.اگر عدد 2 را در نظر بگیریم نمیشود به صورت مجموع دو عدد اول طبیعی نوشت

			\subsubsection*{(ii)}
				مثال نقض :
				اگر $A = \{2, 3\}$و $B = \{2\}$و $C = \{3\}$ آنگاه : $A - (B - C) = \{3\}$ و $(A - B) - C = \emptyset$
				\[A - (B - C) \neq (A - B) - C\]
			
			\subsubsection*{(iii)}
				ابتدا مسئله را به نماد های ریاضی تعمیم میدهیم :\\
				\[\forall x, y \in \mathbb{R} : (\frac{x}{y} > \sqrt{3}) \rightarrow\]
				\[
				\frac{x^2}{y^2} > 3 \rightarrow x^2 > 3y^2 \rightarrow x^2 - 3y^2 > 0
				\]
				اگر $u = 2x + 3y$, $v = x + 2y$ باشد آنگاه داریم :\\
				\[u^2 -3v^2 = x^2 - 3y^2 > 0 \rightarrow \frac{2x + 3y}{x + 2y} > \sqrt{3} \rightarrow\]
				\[x^2 - 3y^2 > 0 \rightarrow x^2 > 3y^2 \rightarrow x^2 + 2xy > 3y^2 + 2xy \rightarrow\]
				\[x(x + 2y) > y(2x + 3y) \rightarrow \frac{x}{y} > \frac{2x + 3y}{x + 2y} \rightarrow\]
				\[\frac{x}{y} > \frac{2x + 3y}{x + 2y} > \sqrt{3}\]
				در نتیجه همیشه یک عدد گویای کوچک تری وجود دارد که از $\sqrt{3}$ بزرگتر باشد.

			\subsubsection*{(iv)}
				\[
				\forall x,y \in \mathbb{R}, \epsilon > 0 : (y \leq x + \epsilon \Rightarrow y \leq x) \rightarrow
				\]
				\[
				\epsilon = \frac{1}{n} \rightarrow y \leq x + \frac{1}{n} \rightarrow \lim_{n \to \infty} y = y, 
				\lim_{n \to \infty} x + \frac{1}{n} = x \Rightarrow
				\]
				\[
				y \leq x
				\]
		
		\subsection*{جواب سوال ششم :}
			\subsubsection*{(i)}
				درست است چون به ازای هر $x$ یک $x - 1$ وجود دارد و چون $x$ > $x - 1$  این گزاره درست است.
			
			\subsubsection*{(ii)}
				مثال نقض :
				اگر $x = 0$ پس باید $y = 3$ به ازای همه مقادیر $y$ که ممکن نیست.
			
			\subsubsection*{(iv)}
				(الف) :\\
				از آنجا که $\emptyset \in P(X)$ هست پس $B$ میتواند مساوی $\emptyset$ باشد.\\
				اگر $B = \emptyset$ باشد، گزاره $\forall A \in P(X) \exists  B \in P(X): (A \cup B = A \equiv A = A)$ پس این گزاره درست است. \\

				(ب) :\\
				از آنجا که $\emptyset \in P(X)$ هست پس $A$ میتواند مساوی $\emptyset$ باشد.\\
				اگر $A = \emptyset$ باشد به ازای همه مقادیر $B$ داریم :\\
				$\exists A \in \emptyset \forall B \in P(X): (A \cup B = B \equiv B = B)$ پس این گزاره درست است.

		\subsection*{جواب سوال هفتم :}
			اگر، تحصیل در رشته پزشکی $p \equiv $, تحصیل در رشته ریاضی $q \equiv$, داشتن درآمد خوب $r \equiv $,\\
			برداشت منطقی از زندگی $t \equiv $, احساس بطالت کردن $z \equiv $\\ باشد آنگاه :
			\[ p \Rightarrow r , q \Rightarrow t , (r \lor t) \Rightarrow \tildevar z, z \mapsto \tildevar (p \lor q) \]
			با توجه به این تحلیل فرض میکنیم که $\tildevar (p \lor q) \equiv \tildevar p \land \tildevar q$ درست باشد پس ارزش $p$و $q$ هر دو نادرست هستند. \\
			پس $p \Rightarrow r$ و $q \Rightarrow t$ هر دو درست هستند پس $r$ و $t$ میتوانند هر مقاداری باشند اما؛ باید گزاره های دیگر هم بررسی شوند. \\
			حال اگر ارزش گزاره $z$ درست باشد و $r$ , $t$ هر دو نادرست باشند ارزش گزاره های $r \lor t \Rightarrow \tildevar z$ و $z$ همگی درست است.\\
			پس بحث زیر معتبر است.
		
		\subsection*{جواب سوال هشتم:}
			\subsubsection*{(i)}
				از آنجا که $(x + y)^n = \sum_{r = 0}^{n} C(n, r) x^{n-r} y^r$ داریم :\\
				\[
				x,y = q \rightarrow (1 + 1)^n = \sum_{r = 0}^{n} C(n, r) 1^{n-r} 1^r \rightarrow
				2^n = \sum_{r = 0}^{n} C(n, r)
				\]
			\subsubsection*{(ii)}
				اگر $P(n) = 1^3 + 2^3 + ... + n^3$ :\\
				1) اگر $n = 1$ : $P(1) = 1$ و $\frac{1^2 (1 + 1)^2}{4} = 1$ پس شرط اول برقرار است.\\
				2) با فرض برقرار بودن $P(k) = \frac{k^2 (k + 1)^2}{4}$ به بررسی برابری $P(k + 1)$ میپردازیم :\\
				\[P(k + 1) = 1^3 + 2^3 + ... k^3 + (k + 1)^3 \rightarrow\]
				\[P(k + 1) = P(k) + (k + 1)^3, P(k) = \frac{k^2 (k + 1)^2}{4} \rightarrow\]
				\[P(k + 1) = \frac{k^2 (k + 1)^2}{4} + (k + 1)^3 = \frac{(k + 1)^2 (k + 2)^2}{4}\]

			\subsubsection*{(iii)}
				اگر $P(n) = \frac{1}{1 \times 2} + \frac{1}{2 \times 3} + ... + \frac{1}{n \times (n + 1)}$ :\\
				1) اگر $n = 1$ : $P(1) = \frac{1}{2}$, $\frac{1}{1 + 1} = \frac{1}{2}$ پس شرط اول برقرار است.\\
				2)با فرض برقرار بودن $P(k) = \frac{k}{k + 1}$ به بررسی برابری $P(k + 1)$ میپردازیم :\\
				\[P(k + 1) = \frac{1}{1 \times 2} + \frac{1}{1 \times 2} + ... + \frac{k}{k \times (k + 1)} + \frac{k + 1}{(k + 1) \times (k + 2)} \rightarrow\]
				\[P(k + 1) = P(k) + \frac{k + 1}{(k + 1) \times (k + 2)}, P(k) = \frac{k}{k + 1} \rightarrow\]
				\[P(k + 1) = \frac{k}{k + 1} + \frac{k + 1}{(k + 1) \times (k + 2)} = \frac{k + 1}{k + 2}\]

		\subsection*{جواب سوال نهم:}
			\subsubsection*{(الف)}
				مجموعه $A$ که شامل 2 نباشد مساوی $\{1, 3, 4, 5, ..., 9\}$ پس تعداد زیر مجموعه های آن : $2^8 = 256$ است.
			
			\subsubsection*{(ب)}
				با استفاده از قانون دمورگان برای محاسبه تعداد زیر مجموعه های $A$ که شامل 2 باشد، مقدار کل زیر مجموعه های $A$ را از تعداد زیر مجموعه های $A$ .که شامل 2 نیستند کم میکنیم
				پس داریم :\\
				\[
					2^9 - 2^8 = 256			
				\]

			\subsubsection*{(ج)}
				برای محسابه تعداد زیر مجموعه هایی که شامل 5 نباشد ولی شامل 7 باشد، ابتدا فرم سوال را به فرم مجموعه تبدیل میکنیم :\\
				\[|5 \not \in A \land 7 \in A| \equiv |5 \not \in A - 7 \not \in A|\]
				این یعنی که از مجموعه که حاوی 5 نیست، 7 را هم کم کن پس داریم : \\
				\[B = 5 \not \in A \rightarrow |5 \not \in A - 7 \not \in A| \equiv |B| - |7 \not \in B| \rightarrow\]
				\[2^8 - 2^7 = 2^7 = 128\]

			\subsubsection*{(د)}
				به فرم سوال بالا حل میشود.

\end{document}